\documentclass[a4paper]{article}
\usepackage[utf8]{inputenc}
\usepackage[russian,english]{babel}
\usepackage[T2A]{fontenc}
\usepackage{polynom}
% this is necessary for \mathbb to work
\usepackage{amsfonts}
% math package
\usepackage{amsmath}
\usepackage{polynomial}


\title{Домашняя работа}
\author{Акентьев Роман}
\date\today

\begin{document}
\maketitle
\section*{Комплексные числа и многочлены}
\subsection*{Многочлены и рациональные функции}

% PROBLEM 188
\subsubsection*{№188. Найдите целые действительные корни многочлена $x^3-5x^2+14x-16$.}
\textbf{Решение:} \par

Согласно теореме о знаках Декарта, у многочлена $x^3-5x^2+14x-16$ есть 1 или 3 положительных корня $x \in \mathbb{R}$. \par
Согласно теореме о рациональных корнях, все рациональные корни многочлена $x=\frac{p}{q}$, где $p$ --делитель свободного члена ($-16$), а $q$-делитель старшего коэффициента ($1$). \par
Рациональные корни-кандидаты многочлена: $\pm16; \pm8; \pm4; \pm2; \pm1$.
\[P(2)=2^3-5\cdot2^2+14\cdot2-16=-20-16+8+28=0 \Rightarrow x_0=2\] \par
Т.к. по теореме Безу остаток от $\frac{P(x)}{x-a}=P(a)$, то мы можем разложить исходный многочлен на множители: 
\[\polylongdiv[style=C]{x^3-5x^2+14x-16}{x-2}\] \par
Решим уравнение $x^2-3x+8=0$: \par
\[x_1,_2=\frac{-(-3)\pm\sqrt{(-3)^2-4\cdot1\cdot8}}{2}=\frac{3\pm\sqrt{-23}}{2}\] \par
Квадратный корень из отрицательного числа не существует в области вещественных чисел: $x \in  \emptyset$. \newline
\textbf{Ответ:} $x=2$

% PROBLEM 189
\subsubsection*{№189. Найдите целые действительные корни многочлена $x^4-2x^3+2x^2-22x+21$.}
\textbf{Решение:} \par

Согласно теореме о знаках Декарта, у многочлена $x^4-2x^3+2x^2-22x+21$ есть 2 или 4 положительных корня $x \in \mathbb{R}$. \par
Согласно теореме о рациональных корнях, все рациональные корни многочлена $x=\frac{p}{q}$, где $p$ --делитель свободного члена ($-21$), а $q$-делитель старшего коэффициента ($1$). \par
Рациональные корни-кандидаты многочлена: $\pm21; \pm7; \pm3; \pm1$.
\[P(1)=1^4-2\cdot1^3+2\cdot1^2-22\cdot1+21=0 \Rightarrow x_0=1\] \par
Т.к. по теореме Безу остаток от $\frac{P(x)}{x-a}=P(a)$, то мы можем разложить исходный многочлен на множители: 
\[\polylongdiv[style=C]{x^4-2x^3+2x^2-22x+21}{x-1}\] \par

Разложим на множители многочлен $x^3-x^2+x-21$: \par
Согласно теореме о рациональных корнях, все рациональные корни многочлена $x=\frac{p}{q}$, где $p$ --делитель свободного члена ($-21$), а $q$-делитель старшего коэффициента ($1$). \par
Рациональные корни-кандидаты многочлена: $\pm21; \pm7; \pm3; \pm1$.
\[P(3)=3^3-3^2-21=-21-9+27+3=0 \Rightarrow x_1=3\] \par
Т.к. по теореме Безу остаток от $\frac{P(x)}{x-a}=P(a)$, то:
\[\polylongdiv[style=C]{x^3-x^2+x-21}{x-3}\] \par
Решим уравнение $x^2+2x+7=0$: \par
\[x_1,_2=\frac{-2\pm\sqrt{2^2-4\cdot1\cdot7}}{2}=\frac{-2\pm\sqrt{-24}}{2}\] \par
Квадратный корень из отрицательного числа не существует в области вещественных чисел: $x \in  \emptyset$. \newline
\textbf{Ответ:} $x_0=1, x_1=3$

% PROBLEM 190
\subsubsection*{№190. Найдите целые действительные корни многочлена $x^5+5x^4+6x^3-x^2-5x-6$.}
\textbf{Решение:} \par

Согласно теореме о знаках Декарта, у многочлена $x^5+5x^4+6x^3-x^2-5x-6$ есть 1 положительный корень $x \in \mathbb{R}$. \par
Разложим многочлен на множители:

\begin{gather*}
	x^5+5x^4+6x^3-x^2-5x-6=(x^5+5x^4+6x^3)-(x^2+5x+6)= \\
	=x^3(x^2+5x+6)-(x^2+5x+6)=(x^2+5x+6)(x^3-1)=\textbf{(*)} \\
	x^2+5x+6=(x^2+2x)+(3x+6)=x(x+2)+3(x+2)=(x+2)(x+3) \\
	x^3-1=x^3-1^3=(x-1)(x^2+x+1) \\
	\textbf{(*)}=(x+2)(x+3)(x-1)(x^2+x+1)
\end{gather*}

Решим уравнение $x^2+x+1=0$: \par
\[x_1,_2=\frac{-1\pm\sqrt{1^2-4\cdot1\cdot1}}{2}=\frac{-1\pm\sqrt{-3}}{2}\] \par
Квадратный корень из отрицательного числа не существует в области вещественных чисел: $x \in  \emptyset$. \newline
\textbf{Ответ:} $x_0=1, x_1=-2, x_3=-3$

% PROBLEM 191
\subsubsection*{№191. Найдите наибольший общий делитель многочленов $x^3-6x^2+11x-6$ и $x^3+3x^2-25x+21$.}
\textbf{Решение:} \par

Применим алгоритм Евклида: \par
\polylongdiv[style=C]{x^3+3x^2-25x+21}{x^3-6x^2+11x-6} \par
\polylongdiv[style=C]{x^3-6x^2+11x-6}{9x^2-36x+27} \par
\polylonggcd{x^3+3x^2-25x+21}{x^3-6x^2+11x-6}
\[9x^2-36x+27=x^2-4x+3\]
\textbf{Ответ:} НОД: $x^2-4x+3$

% PROBLEM 192
\subsubsection*{№192. Найдите наибольший общий делитель многочленов $x^3-9x^2+23x-15$ и $x^3-5x^2-28x+32$.}
\textbf{Решение:} \par

Применим алгоритм Евклида: \par
\polylongdiv[style=C]{x^3-9x^2+23x-15}{x^3-5x^2-28x+32} \par
\polylongdiv[style=C]{x^3-5x^2-28x+32}{-4x^2+51x-47} \par
\polylongdiv[style=C]{-4x^2+51x-47}{\frac{945}{16}x-\frac{945}{16}} \par
\polylonggcd{x^3-9x^2+23x-15}{x^3-5x^2-28x+32}
\[\frac{945}{16}x-\frac{945}{16}=\frac{16}{945}\cdot(\frac{945}{16}x-\frac{945}{16})=x-1\]
\textbf{Ответ:} НОД: $x-1$

% PROBLEM 193
\subsubsection*{№193. В дроби $\frac{8x^3-x^2-10x+2}{x+1}$ выделите целую часть.}
\textbf{Решение:} \par
Разделим многочлен в числителе на многочлен в знаменателе: \par
\polylongdiv[style=C]{8x^3-x^2-10x+2}{x+1} \par
\textbf{Ответ:} $8x^2-9x-1+\frac{3}{x+1}$

% PROBLEM 194
\subsubsection*{№194. Представьте дробь $\frac{10(-2x-9)}{x^2+12x+11}$ в виде суммы простейших дробей над $\mathbb{R}$}
\textbf{Решение:} \par

Применим алгоритм неопределённых коэффициентов. Разложим многочлен в знаменателе на множители:
\[x^2+12x+11=(x^2+11x)(x+11)=x(x+11)+1\cdot(x+11)=(x+1)(x+11)\]
Представим дробь из условия в виде суммы простейших дробей с неопределёнными коэффициентами:
\[\frac{10(-2x-9)}{(x+1)(x+11)}=\frac{A}{x+1}+\frac{B}{x+11}=\frac{A(x+11)+B(x+1)}{(x+1)(x+11)}=\frac{(A+B)x+11A+B}{(x+1)(x+11)}\]
Равенство сводится к равенству:
\[-20x-90=(A+B)x+11A+B\]
По правилу равенства многочленов (многочлены равны тогда и только тогда, когда их коэффициенты при одинаковых степенях совпадают):
\begin{equation*}
 \begin{cases}
   	A+B=-20
   	\\
   	11A+B=-90
   	\\
 \end{cases}
\end{equation*}
$\Rightarrow$
\begin{equation*}
 \begin{cases}
   	A=-B-20
   	\\
   	11(-B-20)+B=-90
   	\\
 \end{cases}
\end{equation*}
$\Rightarrow$
\begin{equation*}
 \begin{cases}
   	A=-(-13)-20=-7
   	\\
   	B=\frac{-90+220}{10}=13
   	\\
 \end{cases}
\end{equation*}
\textbf{Ответ:} $\frac{10(-2x-9)}{x^2+12x+11}=-\frac{7}{(x+1)}-\frac{13}{(x+11)}$

% PROBLEM 195
\subsubsection*{№195. Представьте дробь $\frac{2x^2+13x+37}{(x+6)(x^2+12x+36)}$ в виде суммы простейших дробей над $\mathbb{R}$}
\textbf{Решение:} \par
Очевидно, что $(x+6)(x^2+12x+36)=(x+3)^3$ \par
Применим алгоритм неопределенных коэффициентов: 
\[\frac{2x^2+13x+37}{(x+6)(x^2+12x+36)}=\frac{2x^2+13x+37}{(x+6)^3}=\frac{A}{x+6}+\frac{B}{(x+6)^2}+\frac{C}{(x+6)^3}=\]
\[=\frac{A(x+6)^2+B(x+6)+C}{(x+6)^3}=\frac{Ax^2+(12A+B)+36A+6B+C}{(x+6)^3}\]
Равенство сводится к равенству:
\[2x^2+13x+37=Ax^2+(12A+B)x+36A+6B+C\]
По правилу равенства многочленов:
\begin{equation*}
 \begin{cases}
	A=2 \\
	12A+B=13 \\
	36A+6B+C=37 \\
 \end{cases}
\end{equation*}
$\Rightarrow$
\begin{equation*}
 \begin{cases}
	A=2 \\
	12\cdot2+B=13 \\
	36\cdot2+6B+C=37 \\
 \end{cases}
\end{equation*}
$\Rightarrow$
\begin{equation*}
 \begin{cases}
	A=2 \\
	B = -13-24=-11 \\
	C = 37-72+66=31 \\
 \end{cases}
\end{equation*}
\textbf{Ответ:} $\frac{2x^2+13x+37}{(x+6)(x^2+12x+36)}=\frac{2}{(x+6)}+\frac{-11}{(x+6)^2}+\frac{31}{(x+6)^3}$

% PROBLEM 196
\subsubsection*{№196. Представьте дробь $\frac{2x^2+12x-18}{(x+6)^2(x+8)}$ в виде суммы простейших дробей над $\mathbb{R}$}
\textbf{Решение:} \par 
Применим алгоритм неопределённых коэффициентов:
\[\frac{2x^2+12x-18}{(x+6)^2(x+8)}=\frac{A}{x+6}+\frac{B}{(x+6)^2}+\frac{C}{x+8}=\frac{A(x^2+14x+48)+B(x+8)+C(x+6)^2}{(x+6)^2(x+8)}\]
Равенство сводится к равенству:
\[2x^2+12x-18=(A+C)x^2+(14A+12C+B)x+48A+8B+36C\]
По правилу равенства многочленов:
\begin{equation*}
 \begin{cases}
	A+C=2 \\
	14A+12C+b=12 \\
	48A+8B+36C=-18 \\
 \end{cases}
\end{equation*}
$\Rightarrow$
\begin{equation*}
 \begin{cases}
	C=2-A \\
	14A+12(2-A)+B=12 \\
	48A+8B+36(2-A)=-18 \\
 \end{cases}
\end{equation*}
$\Rightarrow$
\begin{equation*}
 \begin{cases}
	C=2-(-\frac{3}{2})=\frac{7}{2} \\
	B=-2\cdot(-\frac{3}{2})-12=-9 \\
	A=-\frac{3}{2}
 \end{cases}
\end{equation*}
\textbf{Ответ:} $\frac{2x^2+12x-18}{(x+6)^2(x+8)}=\frac{7}{2(x+8)}-\frac{3}{2(x+6)}-\frac{9}{(x+6)^2}$

% PROBLEM 197
\subsubsection*{№197. Представьте дробь $\frac{9(x^2+6x+5)}{(x+4)(x^2+8x+25)}$ в виде суммы простейших дробей над $\mathbb{R}$}
\textbf{Решение:} \par 
Применим алгоритм неопределённых коэффициентов:
\[\frac{9(x^2+6x+5)}{(x+4)(x^2+8x+25)}=\frac{A}{x+4}+\frac{B\cdot x+C}{x^2+8x+25}=\]\[=\frac{A(x^2+8x+25)+(B\cdot x+C)(x+4)}{(x+4)(x^2+8x+25)}=\frac{(A+B)x^2+(8A+4B+C)x+25A+4C}{(x+4)(x^2+8x+25)}\]
Равенство сводится к равенству:
\[9x^2+54x+45=(A+B)x^2+(8A+4B+C)x+25A+4C\]
По правилу равенства многочленов:
\begin{equation*}
 \begin{cases}
	A+B=9 \\
	8A+4B+C=54 \\
	25A+4C=45 \\
 \end{cases}
\end{equation*}
$\Rightarrow$
\begin{equation*}
 \begin{cases}
	A=9-B \\
	8(9-B)+4B+C=54 \\
	25A+4C=45 \\
 \end{cases}
\end{equation*}
$\Rightarrow$
\begin{equation*}
 \begin{cases}
	A=9-B \\
	C=4B-18 \\
	25(9-B)+4(4B-18)=45 \\
 \end{cases}
\end{equation*}
$\Rightarrow$
\begin{equation*}
 \begin{cases}
	A=9-B \\
	C=4B-18 \\
	9B=108 \\
 \end{cases}
\end{equation*}
$\Rightarrow$
\begin{equation*}
 \begin{cases}
	A=-3 \\
	B=12 \\
	C=30 \\
 \end{cases}
\end{equation*}
\textbf{Ответ:} $\frac{9(x^2+6x+5)}{(x+4)(x^2+8x+25)}=\frac{12x+30}{x^2+8x+25}-\frac{3}{x+4}$

% PROBLEM 198
\subsubsection*{№198. Представьте дробь $\frac{(x^2+8x+2)}{(x+4)(x+3)(x+2)}$ в виде суммы простейших дробей над $\mathbb{R}$}
\textbf{Решение:} \par
Применим алгоритм неопределённых коэффициентов:
\[\frac{(x^2+8x+2)}{(x+4)(x+3)(x+2)}=\frac{A}{(x+4)}+\frac{B}{(x+3)}+\frac{C}{(x+2)}=\]
\[=\frac{A(x+3)(x+4)+B(x+2)(x+4)+C(x+2)(x+3)}{(x+4)(x+3)(x+2)}=\]
\[=\frac{A(x^2+7x+12+B(x^2+6x+8)+C(x^2+5x+6)}{(x+4)(x+3)(x+2)}=\]
\[=\frac{(A+B+C)x^2+(7A+6B+5C)x+12A+8B+6C}{(x+4)(x+3)(x+2)}\]
Равенство сводится к равенству:
\[x^2+8x+2=(A+B+C)x^2+(7A+6B+5C)x+12A+8B+6C\]
По правилу равенства многочленов:
\begin{equation*}
 \begin{cases}
	A+B+C=1 \\
	7A+6B+5C=8 \\
	6A+4B+3C=1 \\
 \end{cases}
\end{equation*}
$\Rightarrow$
\begin{equation*}
 \begin{cases}
	C=1-A-B \\
	7A+6B+5(1-A-B)=8 \\
	6A+4B+3(1-A-B)=1 \\
 \end{cases}
\end{equation*}
$\Rightarrow$
\begin{equation*}
 \begin{cases}
	C=1-A-B \\
	7A+6B+5-5A-5B=8 \\
	6A+4B+3-3A-3B=1 \\
 \end{cases}
\end{equation*}
$\Rightarrow$
\begin{equation*}
 \begin{cases}
	C=1-A-B \\
	2A+B=3 \\
	3A+B=-2 \\
 \end{cases}
\end{equation*}
$\Rightarrow$
\begin{equation*}
 \begin{cases}
	C=1-A-(3-2A) \\
	B=3-2A \\
	3A+(3-2A)=-2 \\
 \end{cases}
\end{equation*}
$\Rightarrow$
\begin{equation*}
 \begin{cases}
	C=1-A-(3-2A) \\
	B=3-2A \\
	A=-5 \\
 \end{cases}
\end{equation*}
$\Rightarrow$
\begin{equation*}
 \begin{cases}
	C=-7 \\
	B=13 \\
	A=-5 \\
 \end{cases}
\end{equation*}
\textbf{Ответ:} $\frac{(x^2+8x+2)}{(x+4)(x+3)(x+2)}=\frac{13}{x+3}-\frac{7}{x+4}-\frac{5}{x+2}$

% PROBLEM 199
\subsubsection*{№199. Представьте дробь $\polynomialfrac[reciprocal]{1,4,-11,-44,-8,16,-18}{1,2,-13,-14,24}$ в виде суммы многочлена и простейших дробей над $\mathbb{R}$}
\textbf{Решение:} \par
Разделим числитель на знаменатель: \par
\[\polylongdiv[style=A]{x^6+4x^5-11x^4-44x^3-8x^2+16x-18}{x^4+2x^3-13x^2-14x+24}\] \par
\[\frac{x^6+4x^5-11x^4-44x^3-8x^2+16x-18}{x^4+2x^3-13x^2-14x+24}=\frac{-30x^2-60x+30}{x^4+2x^3-13x^2-14x+24}+x^2+2x-2\]

Разложим на множители многочлен в знаменателе $x^4+2x^3-13x^2-14x+24$: \par
Согласно теореме о знаках Декарта, многочлен имеет 2 положительных корня $x \in \mathbb{R}$. \par
Согласно теореме о рациональных корнях, все рациональные корни многочлена $x=\frac{p}{q}$, где $p$ --делитель свободного члена ($24$), а $q$ -- делитель старшего коэффициента ($1$). \par
Рациональные корни-кандидаты многочлена: $\pm24; \pm12; \pm8; \pm6; \pm4; \pm2; \pm1$
\[P(1)=1^4+2\cdot1^3-13\cdot1^2-14\cdot1+24=-13-14+1+2+24=-27+27=0 \Rightarrow x_0=1\]
Т.к. по теореме Безу остаток от $\frac{P(x)}{x-a}=P(a)$, то мы можем разложить многочлен таким образом: \par
\polylongdiv[style=C]{x^4+2x^3-13x^2-14x+24}{x-1}

Разложим на множители многочлен $x^3+3x^2-10x-24$: \par
Согласно теореме о знаках Декарта, многочлен имеет 2 положительных корня $x \in \mathbb{R}$. \par
Согласно теореме о рациональных корнях, все рациональные корни многочлена $x=\frac{p}{q}$, где $p$ --делитель свободного члена ($-24$), а $q$ -- делитель старшего коэффициента ($1$). \par
Рациональные корни-кандидаты многочлена: $\pm24; \pm12; \pm8; \pm6; \pm4; \pm2; \pm1$
\[P(-2)=(-2)^3+3(-2)^2-10(-2)-24=-8-24+12+20=-32+32=0 \Rightarrow x_1=-2\]
Т.к. по теореме Безу остаток от $\frac{P(x)}{x-a}=P(a)$, то мы можем разложить многочлен таким образом: \par
\polylongdiv[style=C]{x^3+3x^2-10x-24}{x+2}

Разложим на множители многочлен $x^2+x-12$:
\[x^2+x-12=(x^2-3x)+(4x-12)=x(x-3)+4(x-3)=(x+4)(x-3)\]

В результате имеем:
\[x^4+2x^3-13x^2-14x+24=\polyfactorize{x^4+2x^3-13x^2-14x+24}\]
Воспользуемся алгоритмом неопределенных коэффициентов:
\[\frac{-30x^2-60x+30}{(x-1)(x+2)(x+4)(x-3)}=\frac{A}{x-3}+\frac{B}{x-1}+\frac{C}{x+2}+\frac{D}{x+4}=\]
\[=\frac{A(x-1)(x+2)(x+4)+B(x-3)(x+4)(x+2)+C(x-3)(x-1)(x+4)+D(x-3)(x-1)(x+2)}{\polyfactorize{x^4+2x^3-13x^2-14x+24}}=\]
\[=\frac{(A+B+C+D)x^3+(5A+3B-2D)x^2+(2A-10B-13C-5D)x-8A-24B+12C+6D}{\polyfactorize{x^4+2x^3-13x^2-14x+24}}\]
Равенство сводится к равенству:
\[-30x^2-60x+30=(A+B+C+D)x^3+(5A+3B-2D)x^2+(2A-10B-13C-5D)x-8A-24B+12C+6D\]
По правилу равенства многочленов:
\begin{equation*}
 \begin{cases}
	A+B+C+D=0 \\
	5A+3B-2D=-30 \\
	2A-10B-13C-5D=-60 \\
	-8A-24B+12C+6D=30 \\
 \end{cases}
\end{equation*}
Применим метод Гаусса для решения системы: \par
\[
\left[
\begin{array}{cccc|c}
	1 & 1 & 1 & 1 & 0 \\ 
	5 & 3 & 0 & -2 & -30 \\
	2 & -10 & -13 & -5 & -60\\
	-8 & -24 & 12 & 6 & 30\\ 
\end{array}
\right]
\to
\left[
\begin{array}{cccc|c}
	1 & 1 & 1 & 1 & 0 \\ 
	0 & -12 & -15 & -7 & -60 \\
	5 & 3 & 0 & -2 & -30 \\
	-8 & -24 & 12 & 6 & 30 \\ 
\end{array}
\right]
\to
\left[
\begin{array}{cccc|c}
	1 & 1 & 1 & 1 & 0 \\ 
	0 & -12 & -15 & -7 & -60 \\
	0 & -2 & -5 & -7 & -30 \\
	-8 & -24 & 12 & 6 & 30 \\ 	
\end{array}
\right]
\to\]
\[
\left[
\begin{array}{cccc|c}
	1 & 1 & 1 & 1 & 0 \\ 
	0 & -12 & -15 & -7 & -60 \\
	0 & -2 & -5 & -7 & -30 \\
	0 & -16 & 20 & 14 & 30 \\ 		
\end{array}
\right]
\to
\left[
\begin{array}{cccc|c}
	1 & 1 & 1 & 1 & 0 \\ 
	0 & -16 & 20 & 14 & 30 \\ 
	0 & 0 & -\frac{15}{2} & -\frac{35}{4} & -\frac{135}{4} \\
	0 & -12 & -15 & -7 & -60 \\		
\end{array}
\right]
\to\]
\[
\left[
\begin{array}{cccc|c}
	1 & 1 & 1 & 1 & 0 \\ 
	0 & -16 & 20 & 14 & 30 \\ 
	0 & 0 & -30 & -\frac{35}{2} & -\frac{165}{2} \\		
	0 & 0 & -\frac{15}{2} & -\frac{35}{4} & -\frac{135}{4} \\	
\end{array}
\right]
\to
\left[
\begin{array}{cccc|c}
	1 & 1 & 1 & 1 & 0 \\ 
	0 & -16 & 20 & 14 & 30 \\ 
	0 & 0 & -30 & -\frac{35}{2} & -\frac{165}{2} \\		
	0 & 0 & 0 & -\frac{35}{8} & -\frac{105}{8} \\		
\end{array}
\right]
\to
\left[
\begin{array}{cccc|c}
	1 & 1 & 1 & 1 & 0 \\ 
	0 & -16 & 20 & 14 & 30 \\ 
	0 & 0 & -30 & -\frac{35}{2} & -\frac{165}{2} \\		
	0 & 0 & 0 & 1 & 3 \\		
\end{array}
\right]
\to\]
\[
\left[
\begin{array}{cccc|c}
	1 & 1 & 1 & 1 & 0 \\ 
	0 & -16 & 20 & 14 & 30 \\ 
	0 & 0 & -30 & 0 & -30 \\		
	0 & 0 & 0 & 1 & 3 \\			
\end{array}
\right]
\to
\left[
\begin{array}{cccc|c}
	1 & 1 & 1 & 0 & -3 \\ 
	0 & -16 & 20 & 0 & -12 \\ 
	0 & 0 & 1 & 0 & 1 \\		
	0 & 0 & 0 & 1 & 3 \\			
\end{array}
\right]
\to
\left[f
\begin{array}{cccc|c}
	1 & 1 & 0 & 0 & -4 \\ 
	0 & -16 & 0 & 0 & -32 \\ 
	0 & 0 & 1 & 0 & 1 \\		
	0 & 0 & 0 & 1 & 3 \\		
\end{array}
\right]
\to\]
\[
\left[
\begin{array}{cccc|c}
	1 & 1 & 0 & 0 & -4 \\ 
	0 & 1 & 0 & 0 & 2 \\ 
	0 & 0 & 1 & 0 & 1 \\		
	0 & 0 & 0 & 1 & 3 \\		
\end{array}
\right]
\to
\left[
\begin{array}{cccc|c}
	1 & 0 & 0 & 0 & -6 \\ 
	0 & 1 & 0 & 0 & 2 \\ 
	0 & 0 & 1 & 0 & 1 \\		
	0 & 0 & 0 & 1 & 3 \\			
\end{array}
\right]
\]
\textbf{Ответ:} $\polynomialfrac[reciprocal]{1,4,-11,-44,-8,16,-18}{1,2,-13,-14,24}=\polynomial[reciprocal]{1,2,-2}+\polynomialfrac[reciprocal]{2}{x-1}+\polynomialfrac[reciprocal]{1}{x+2}+\polynomialfrac[reciprocal]{3}{x+4}-\polynomialfrac[reciprocal]{6}{x-3}$

% PROBLEM 200
\subsubsection*{№200. Представьте дробь $\polynomialfrac[reciprocal]{-1,0,2,5,-2,11,-2}{1,1,-2,-16,16}$ в виде суммы многочлена и простейших дробей над $\mathbb{R}$}
\textbf{Решение:} \par
Разделим многочлен в числителе на многочлен в знаменателе:
\[\polylongdiv[style=A]{-x^6+2x^4+5x^3-2x^2+11x-2}{x^4+x^3-2x^2-16x+16}\] \par
Разложим на множители многочлен в знаменателе: \par
Согласно теореме о знаках Декарта, у многочлена $x^4+x^3-2x^2-16x+16$ есть 2 положительных корня $x \in \mathbb{R}$. \par
Согласно теореме о рациональных корнях, все рациональные корни многочлена $x=\frac{p}{q}$, где $p$ --делитель свободного члена ($16$), а $q$-делитель старшего коэффициента ($1$). \par
Рациональные корни-кандидаты многочлена: $\pm16; \pm8; \pm4; \pm2; \pm1$.
\[P(2)=2^4+2^3-2\cdot2^2-16\cdot2+16=-8-16-16+16+16+8=0 \Rightarrow x_0=2\]
Т.к. по теореме Безу остаток от $\frac{P(x)}{x-a}=P(a)$, то мы можем разложить $x^4+x^3-2x^2-16x+16$ на множители: 
\[\polylongdiv[style=C]{x^4+x^3-2x^2-16x+16}{x-2}\] \par
Продолжим разложение $x^3-3x^2+4x-8$ по аналогии:
\[P(1)=1^3+3\cdot1^2+4\cdot1-8=8-8=0 \Rightarrow x_1=1\]
\[\polylongdiv[style=C]{x^3+3x^2+4x-8}{x-1}\] \par
Представим полученную после деления числителя на знаменатель правильную дробь в виде суммы простейших дробей над $\mathbb{R}$:
\[\frac{=8x^3+28x^2-21x+14}{(x-2)(x-1)(x^2+4x+8)}=\frac{A}{x-2}+\frac{B}{x-1}+\frac{C\cdot x+D}{x^2+4x+8}=\]
\[=\frac{A(x-1)(x^2+4x+8)+B(x-2)(x^2+4x+8)+(C\cdot x + D)(x-2)(x-1)}{(x-2)(x-1)(x^2+4x+8)}=\]
\[=\frac{A(x^3+4x^2+8x-x^2-4x-8)+B(x^3+4x^2+8x-2x^2-8x-16)+(C\cdot x+D)(x^2-x-2x+2)}{(x-2)(x-1)(x^2+4x+8)}=\]
\[=\frac{(A+B+C)x^3+(3A+2B-3C+D)x^2+(4A+2C-3D)x-8A-16B+2D}{(x-2)(x-1)(x^2+4x+8)}\]
Равенство сводится к равенству:
\[8x^3+28x^2-21x+14=(A+B+C)x^3+(3A+2B-3C+D)x^2+(4A+2C-3D)x-8A-16B+2D\]
По правилу равенства многочленов:
\begin{equation*}
 \begin{cases}
	A+B+C=-8 \\
	3A-2B-3C+D=28 \\
	4A+2C-3D=-21 \\
	-8A-16B+2D=14
 \end{cases}
\end{equation*}
Применим метод Гаусса для решения системы: \par
\[
\left[
\begin{array}{cccc|c}
	1 & 1 & 1 & 0 & -8 \\
	4 & 0 & 2 & -3 & -21 \\
	3 & 2 & -3 & 1 & 28 \\
	-8 & -16 & 0 & 2 & 14 \\
\end{array}
\right]
\to
\left[
\begin{array}{cccc|c}
	1 & 1 & 1 & 0 & -8 \\
	1 & -4 & 2 & -3 & 11 \\
	3 & 2 & -3 & 1 & 28 \\
	-8 & -16 & 0 & 2 & 14 \\	
\end{array}
\right]
\to
\left[
\begin{array}{cccc|c}
	1 & 1 & 1 & 0 & -8 \\
	1 & -4 & 2 & -3 & 11 \\
	0 & -1 & -6 & 1 & 52 \\
	-8 & -16 & 0 & 2 & 14 \\		
\end{array}
\right]
\to\]
\[
\left[
\begin{array}{cccc|c}
	1 & 1 & 1 & 0 & -8 \\
	1 & -4 & 2 & -3 & 11 \\
	0 & -1 & -6 & 1 & 52 \\
	0 & -8 & 8 & 2 & -50 \\			
\end{array}
\right]
\to
\left[
\begin{array}{cccc|c}
	1 & 1 & 1 & 0 & -8 \\
	1 & -4 & 2 & -3 & 11 \\
	0 & 0 & 22 & -7 & -197 \\
	0 & -8 & 8 & 2 & -50 \\			
\end{array}
\right]
\to
\left[
\begin{array}{cccc|c}
	1 & 1 & 1 & 0 & -8 \\
	0 & -1 & -6 & 1 & 52 \\
	0 & 0 & 56 & -6 & -466 \\
	0 & 0 & 22 & -7 & -197 \\		
\end{array}
\right]
\to\]
\[
\left[
\begin{array}{cccc|c}
	1 & 1 & 1 & 0 & -8 \\
	0 & -1 & -6 & 1 & 52 \\
	0 & 0 & 56 & -6 & -466 \\
	0 & 0 & 22 & -\frac{65}{14} & -\frac{195}{14} \\		
\end{array}
\right]
\to
\left[
\begin{array}{cccc|c}
	1 & 1 & 1 & 0 & -8 \\
	0 & -1 & -6 & 1 & 52 \\
	0 & 0 & 56 & -6 & -466 \\
	0 & 0 & 0 & 1 & 3 \\		
\end{array}
\right]
\to
\left[
\begin{array}{cccc|c}
	1 & 1 & 1 & 0 & -8 \\
	0 & -1 & -6 & 1 & 52 \\
	0 & 0 & 56 & -6 & -448 \\
	0 & 0 & 0 & 1 & 3 \\			
\end{array}
\right]
\to\]
\[
\left[
\begin{array}{cccc|c}
	1 & 1 & 1 & 0 & -8 \\
	0 & -1 & -6 & 0 & 49 \\
	0 & 0 & 56 & 0 & -448 \\
	0 & 0 & 0 & 1 & 3 \\				
\end{array}
\right]
\to
\left[
\begin{array}{cccc|c}
	1 & 1 & 1 & 0 & -8 \\
	0 & -1 & -6 & 0 & 49 \\
	0 & 0 & 1 & 0 & -8 \\
	0 & 0 & 0 & 1 & 3 \\			
\end{array}
\right]
\to
\left[
\begin{array}{cccc|c}
	1 & 1 & 1 & 0 & -8 \\
	0 & -1 & 0 & 0 & 1 \\
	0 & 0 & 1 & 0 & -8 \\
	0 & 0 & 0 & 1 & 3 \\			
\end{array}
\right]
\to\]
\[
\left[
\begin{array}{cccc|c}
	1 & 1 & 0 & 0 & -8 \\
	0 & -1 & 0 & 0 & 1 \\
	0 & 0 & 1 & 0 & -8 \\
	0 & 0 & 0 & 1 & 3 \\				
\end{array}
\right]
\to
\left[
\begin{array}{cccc|c}
	1 &  & 0 & 0 & 1 \\
	0 & 1 & 0 & 0 & -1 \\
	0 & 0 & 1 & 0 & -8 \\
	0 & 0 & 0 & 1 & 3 \\			
\end{array}
\right]
\]
\textbf{Ответ:} $\polynomialfrac[reciprocal]{-1,0,2,5,-2,11,-2}{1,1,-2,-16,16}=\polynomial[reciprocal]{-1,1,-1}+\frac{3-8x}{x^2+4x+8}+\frac{1}{x-2}-\frac{1}{x-1}$ 

\subsection*{Вычисления}
% PROBLEM 201
\subsubsection*{№201. Вычислите значение выражения $\frac{3i+4}{5i-3}$ и представьте результат в виде $a+bi$.}
\textbf{Решение:}
\[\frac{3i+4}{5i-3}=\frac{(3i+4)(5i+3)}{(5i-3)(5i+3)}=\frac{15i^2+29i+12}{25i^2-9}=\frac{3}{34}-\frac{29}{34}i\]
\textbf{Ответ:} $\frac{3}{34}-\frac{29}{34}i$

% PROBLEM 202
\subsubsection*{№202. Вычислите значение выражения $\frac{(-3+5i)(6+i)}{2+i}$ и представьте результат в виде $a+bi$.}
\textbf{Решение:}
\[\frac{(-3+5i)(6+i)}{2+i}=\frac{(-3+5i)(6+i)(2-i)}{(2+i)(2-i)}=\frac{(5i-3)(12-4i-i^2)}{-i^2+4}=\]
\[=\frac{77i-20i^2-39}{5}=-\frac{19}{5}+\frac{77i}{5}=-3,8+15,4i\]
\textbf{Ответ:}  $-3,8+15,4i$

% PROBLEM 203
\subsubsection*{№203. Вычислите значение многочлена $P(z)=(-1+5i)z^2+(6i-6i)z+(5-2i)$ в точке $z=-4+6i$. }
\textbf{Решение:}
\[P(-4+6i)=(-1+5i)(36i^2-48i+16)+(-24+36i+24i-36i^2)+5-2i=\]
\[=36+48i-16-180i+240+80i-24+36i+24i+36+5-2i=277+6i\]
\textbf{Ответ:} $277+6i$

% PROBLEM 204
\subsubsection*{№204. Пусть $z_1=-1-i, z_2=3-4i$. Вычислите $\displaystyle \frac{z_1}{\overline{z_2}}-\frac{\overline{z_2}}{z1}$}
\textbf{Решение:}
\[\frac{z_1}{\overline {z_2}}-\frac{\overline {z_2}}{z1}=\frac{-1-i}{3+4i}-\frac{3+4i}{-1-i}=\frac{(-1-i)(-1-i)-(3+4i)(3+4i)}{(3+4i)(-1-i)}=\]
\[=\frac{1+i+i+i^2-(9+24i+16i^2)}{-3-3i-4i-4i^2}=\frac{7-22i}{1-7i}=\frac{(7-22i)(1+7i)}{(1-7i)(1+7i)}=\]
\[=\frac{7+45i-22i-154i^2}{1+7i-7i-49i^2}=\frac{161+27i}{50}=\frac{161}{50}+\frac{27}{50}i=3,22+0,54i\]
\textbf{Ответ:} $3,22+0,54i$

% PROBLEM 205
\subsubsection*{№205. Пусть $z_1=1-3i, z_2=2-i$. Вычислите $\displaystyle \frac{\overline {z_1}+z_2}{z_1-\overline{z_2}}$}
\textbf{Решение:}
\[\frac{\overline {z_1}+z_2}{z_1-\overline{z_2}}=\frac{1+3i+2-i}{1-3i-2-i}=\frac{3+2i}{-1-4i}=\frac{(3+2i)(-1+4i)}{(-1-4i)(-1+4i)}=\]
\[=\frac{-3+12i-2i+8i^2}{1-16i^2}=\frac{-11+10i}{17}=-\frac{11}{17}+\frac{10}{17}i\] 
\textbf{Ответ:} $-\frac{11}{17}+\frac{10}{17}i$

\subsection*{Модуль и аргумент комплексного числа}
% PROBLEM 206
\subsubsection*{№206. Вычислите модуль и аргумент числа $z=-2-2\sqrt{3i}$}
\textbf{Решение:}
\[r=|a+bi|=\sqrt{a^2+b^2}\]
\[a<0, b<0 \Rightarrow Arg(z) = -\pi+\arctan{\frac{b}{a}}\]
\[r=|-2-2\sqrt{3i}|=\sqrt{(-2)^2+(-2\sqrt{3})^2}=\sqrt{4+12}=4\]
\[Arg(z)=-\pi+\arctan{\frac{-2\sqrt{3}}{-2}}=\frac{-2\pi}{3}\]
\textbf{Ответ:} $\displaystyle r=4; Arg(z)=\frac{-2\pi}{3}$

% PROBLEM 207
\subsubsection*{№207. Пусть $u=2(\cos{\frac{\pi}{4}}+i\sin{\frac{\pi}{4}}), v=3(\cos{\frac{\pi}{2}}+i\sin{\frac{\pi}{2}}$). Найдите модуль и аргумент числа $\displaystyle z=\frac{u^5}{\overline{v^6}}$.}
\textbf{Решение:}
\[u=2(\cos{\frac{\pi}{4}}+i\sin{\frac{\pi}{4}})=\frac{2}{\sqrt{2}}+\frac{2}{\sqrt{2}i}=\frac{2+2i}{\sqrt{2}}=\sqrt{2}+\sqrt{2}i\]
\[u^5=(\sqrt{2}+\sqrt{2}i)^5=(-16-16i)\sqrt{2}\]
\[v=3(\cos{\frac{\pi}{2}}+i\sin{\frac{\pi}{2}}=3(0+i\cdot1)=3i\]
\[\overline{v}=-3i\]
\[(\overline{v})^6=(-3i)^6=-729\]
\[z=\frac{(-16-16i)\sqrt{2}}{-729}=\frac{16\sqrt{2}+16\sqrt{2}\cdot i}{729}=\frac{16\sqrt{2}}{729}+\frac{16\sqrt{2}}{729}i\]
\[r=\sqrt{(\frac{16\sqrt{2}}{729})^2+(\frac{16\sqrt{2}}{729})^2}=\sqrt{\frac{512}{729^2}+\frac{512}{729^2}}=\sqrt{\frac{1024}{729^2}}=\frac{32}{729}\]
\[a>0 \Rightarrow Arg(z)=\arctan{\frac{\frac{16\sqrt{2}}{729}}{\frac{16\sqrt{2}}{729}}}=\arctan{1}=\frac{\pi}{4}\]
\textbf{Ответ:} $\displaystyle r=\frac{32}{729}; Arg(z)=\frac{\pi}{4}$

% PROBLEM 208
\subsubsection*{№208. Пусть $u=3(\cos{\frac{\pi}{6}}+i\sin{\frac{\pi}{6}}), v=4(\cos{\frac{\pi}{7}}+\sin{\frac{\pi}{7}})$. Найдите модуль и аргумент числа $\displaystyle z=\overline{u}^4\cdot\overline{v}^5$. Значение аргумента укажите на отрезке $[0;2\pi)$.}
\textbf{Решение:}
\[\overline{u}^4=(3(\cos{-\frac{\pi}{6}}+i\sin{-\frac{\pi}{6}}))^4=81(\cos{-\frac{2\pi}{3}}+i\sin{-\frac{2\pi}{3}})\]
\[\overline{v}^5=1024(\cos{-\frac{5\pi}{7}}+i\sin{-\frac{5\pi}{7}})\]
\[z=81\cdot1024\cdot(\cos{(-\frac{2\pi}{3}+(-\frac{5\pi}{7}))}i\sin{(-\frac{2\pi}{3}+(-\frac{5\pi}{7}))})=\]
\[=82944(\cos{-\frac{29\pi}{21}}+i\sin{-\frac{29\pi}{21}})\]
\textbf{Ответ:} $\displaystyle r=82944; Arg(z)=-\frac{29\pi}{21}$

% PROBLEM 209
\subsubsection*{№209. Приведите число $z=-1+\sqrt{3}i$ к тригонометрическому виду.}
\textbf{Решение:}
\[|z|=\sqrt{(-1)^2+(\sqrt{3})^2}=2\]
\[a<0, b>0 \Rightarrow Arg(z) = \pi + \arctan(-\sqrt{3})=\pi-\arctan\sqrt{3}=\pi-\frac{\pi}{3}=\frac{2\pi}{3}\]
\[z=2(\cos{\frac{2\pi}{3}}+i\sin{\frac{2\pi}{3}})\]
\textbf{Ответ:} $z=2(\cos{\frac{2\pi}{3}}+i\sin{\frac{2\pi}{3}})$

% PROBLEM 210
\subsubsection*{№210. Приведите число $z=4(\cos{\pi}+i\sin{\pi})$ к алгебраическому виду.}
\textbf{Решение:}
\[4(\cos{\pi}+i\sin{\pi}0=4(-1+i\cdot0)=-4\] 
\textbf{Ответ:} $z=-4$

\subsection*{Уравнения}
% PROBLEM 211
\subsubsection*{№211. Найдите комплексные корни уравнения $z^6=64$.}
\textbf{Решение:} \par 
Все корни данного уравнения являются значениями корня $6$ степени из комплексного числа $64+0i$.
Из формулы Муавра следует, что:
\[z=\sqrt[n]{z}=\sqrt[n]{r}(\cos{(\psi)}+i\sin{(\psi)}), \psi=\frac{\varphi+2\pi k}{n}\]
где $r=|z|$ -- модуль комплексного числа, $\varphi=Arg(z)$ -- главное значение аргумента, $n$ -- степень корня, $k$ -- параметр, принимающий значения: $k={0,1,2,3,...,n-1}$.
\[r=|z|=\sqrt{64^2+0^2}=64\]
\[\varphi=\arctan{\frac{0}{64}}=0\]
\[\psi=\frac{0+2\pi k}{6}=\frac{\pi k}{3}\]
\[z=\sqrt[6]{64}(\cos{\psi}+i\sin{\psi}), \psi=\frac{\pi k}{3}\]
\[z_1=2(\cos(0)+i\sin(0))=2\cdot1+i\cdot0=2\]
\[z_2=2(\cos(\frac{\pi}{3}+i\sin{\frac{\pi}{3}})=2\cdot\frac{1}{2}+2i\cdot\frac{\sqrt{3}}{2}=1+\sqrt{3}i\]
\[z_3=2(\cos{\frac{2\pi}{3}}i\sin{\frac{2\pi}{3}})=2\cdot(-\frac{1}{2})+2i\cdot\frac{\sqrt{3}}{2}=-1+\sqrt{3}i\]
\[z_4=2(\cos{\pi}i\sin{\pi})=2\cdot(-1)+2\cdot(i\cdot0)=-2\]
\[z_5=2(\cos{\frac{4\pi}{3}}i\sin{\frac{4\pi}{3}})=2\cdot(-\frac{1}{2})+2i\cdot(-\frac{\sqrt{3}}{2})=-1-\sqrt{3}i\]
\[z_6=2(\cos{\frac{5\pi}{3}}i\sin{\frac{5\pi}{3}})=2(\frac{1}{2})+2i\cdot(-\frac{sqrt{3}}{2})=1-\sqrt{3}i\]
\textbf{Ответ:} $\displaystyle -2;2;1+\sqrt{3}i;-1+\sqrt{3}i;-1-\sqrt{3}i;1-\sqrt{3}i$

% PROBLEM 212
\subsubsection*{№212. Найдите комплексные корни уравнения $x^2+4x+20=0$.}
\textbf{Решение:}
\[x=\frac{-4\pm\sqrt{4^2-4\cdot1\cdot20}}{2\cdot1}=\frac{-4\pm\sqrt{-64}}{2}\]
\[\frac{-4\pm\sqrt{-64}}{2}=\frac{-4\pm8i}{2}=-2\pm4i\]
\textbf{Ответ:} $x_0=-2+4i; x_1=-2-4i$

% PROBLEM 213
\subsubsection*{№213. Найдите комплексные корни уравнения $(1+4i)z^2+(18-13i)z-21-33i=0$.}
\textbf{Решение:}
\[x=\frac{-18+13i\pm\sqrt{(18-13i)^2-4(1+4i)(-21-33i)}}{2+8i}=\]
\[=\frac{-18+13i\pm\sqrt{324-468i-169+84+468i-528}}{2+8i}=\]
\[=\frac{-18+13i\pm\sqrt{-289}}{2+8i}=\frac{-18+13i\pm17i}{2+8i}\]
\[x_0=\frac{-18+30i}{2+8i}=\frac{6(-3+5i)}{2(1+4i)}=\frac{3(-3+5i)(1-4i)}{17}=\frac{3(-3+12i+5i+20}{17}=\frac{3\cdot17(1+1i}{17}=3+3i\]
\[x_1=\frac{-18-4i}{2+8i}=\frac{2(-9-2i}{2(1-4i)}=\frac{(-9-2i)(1-4i)}{17}=\frac{-17(1-2i)}{17}=-1+2i\]
\textbf{Ответ:} $\displaystyle x_0=3+3i; x_1=-1-2i$

% PROBLEM 214
\subsubsection*{№214. Составьте квадратное уравнение с действительными коэффициентами, одним из корней которого является число $z=-1-i$. Сколько существует таких уравнений?}
\textbf{Решение:} \par
Необходимо составить уравнение вида $ax^2+bx+c=0, x_0=-1-i$. \newline
Согласно теореме Виета, $x_0+x_1=-\frac{b}{a}; x_0x_1=\frac{c}{a}$. \newline
Для того, чтобы коэффициенты были действительными, $x_1$ должен быть сопряжённым $x_0$: $x_0=\overline{x_1}$. \newline
Таким образом, $x_1=\overline{-1+i}=-1-i$. \newline
В результате получим систему:
\begin{equation*}
 \begin{cases}
	-1+i-1-i=-\frac{b}{a} \\
	(-1+i)(-1-i)=\frac{c}{a}
 \end{cases}
\Rightarrow
 \begin{cases}
	-2=-\frac{b}{a} \\
	2=\frac{c}{a}
 \end{cases}
\Rightarrow
 \begin{cases}
	2=\frac{b}{a} \\
	2=\frac{c}{a}
 \end{cases}
\Rightarrow
 \begin{cases}
	b=2a \\
	c=2a
 \end{cases}
\Rightarrow
2a=b=c
\end{equation*}
Составим и решим уравнение, удовлетворяющее данному условию:
\[x^2+2x+2=0\]
\[x=\frac{-2\pm\sqrt{2^2-4\cdot1\cdot2}}{2\cdot1}=\frac{-2\pm2i}{2}\]
\[x_0=\frac{-2+2i}{2}=-1+i\]
\[x_1=\frac{-2-2i}{2}=-1-i\]
Для данных корней существует $\infty$ число уравнений с действительными коэффициентами, удовлетворяющими условию $2a=b=c$.\newline
\textbf{Ответ:} $x^2+2x+2=0; \infty$ уравнений с коэффициентами, удовлетворяющими условию $2a=b=c.$
% PROBLEM 215
\subsubsection*{№215. Найдите все рациональные корни уравнения $z^4+8z^3-2z^2-124z-240=0$, если известно, что $z_1=-3-i$ -- один из его корней.}
\textbf{Решение:} \par 
Согласно теореме о знаках Декарта, у многочлена $z^4+8z^3-2z^2-124z-240$ есть 1 положительный корень $x \in \mathbb{R}$. \par
Согласно теореме о рациональных корнях, все рациональные корни многочлена $x=\frac{p}{q}$, где $p$ -- делитель свободного члена ($-240$), а $q$ -- делитель старшего коэффициента ($1$). \par
Рациональные корни-кандидаты многочлена: 
\[\pm240; \pm120; \pm80; \pm60; \pm48; \pm40; \pm30; \pm24; \pm20; \pm16; \pm15; \pm12; \pm10; \pm8; \pm6; \pm5; \pm4; \pm3; \pm2; \pm1\]
\[P(4)=4^4+8\cdot4^3-2\cdot4^2-124\cdot4-240=256+512-32-496-240=0 \Rightarrow z_2=4\]
Т.к. по теореме Безу остаток от $\frac{P(x)}{x-a}=P(a)$, то мы можем разложить исходный многочлен на множители: 
\[\polylongdiv[style=C, vars=z]{z^4+8z^3-2z^2-124z-240}{z-4}\]
Рассмотрим многочлен $z^3+12z^2+46z+60$: \newline
Согласно теореме о рациональных корнях, все рациональные корни многочлена $x=\frac{p}{q}$, где $p$ -- делитель свободного члена ($60$), а $q$ -- делитель старшего коэффициента ($1$). \par
Рациональные корни-кандидаты многочлена: 
\[\pm60; \pm48; \pm40; \pm30; \pm24; \pm20; \pm16; \pm15; \pm12; \pm10; \pm8; \pm6; \pm5; \pm4; \pm3; \pm2; \pm1\]
\[P(-6)=(-6)^3+12(-6)^2+46\cdot(-6)+60=432+60-216-276=0 \Rightarrow z_3=-6\]
Т.к. по теореме Безу остаток от $\frac{P(x)}{x-a}=P(a)$, то мы можем разложить многочлен на множители: 
\[\polylongdiv[style=C, vars=z]{z^3+12z^2+46z+60}{z+6}\]
Решим уравнение $z^2+6z+10=0$:
\[z=\frac{-6\pm\sqrt{(-6)^2-4\cdot1\cdot10}}{2}=\frac{-6\pm\sqrt{-4}}{2}=\frac{-6\pm2i}{2}\]
\[z_1=\frac{-6+2i}{2}=-3+i\]
\[z_4=\frac{-6-2i}{2}=-3-i\]
Также, если у квадратного уравнения $D<0$, то его корни являются комплексно-сопряженными. Т.к. в условии было известно, что $z_1=-3-i$, тогда $z_4=\overline{z_0}=-3+i$. \newline
\textbf{Ответ:} $\displaystyle z_1=-3-i; z_2=4; z_3=-6; z_4=-3+i$
\end{document}